\documentclass{article}
\input{../config/commontex}

\title{Installation of a local ACNUC socket server on your machine (under development)}
\author{Penel, S.}

\usepackage{/Library/Frameworks/R.framework/Resources/share/texmf/Sweave}
\begin{document}
%
% To change the R input/output style:
%
\definecolor{Soutput}{rgb}{0,0,0.56}
\definecolor{Sinput}{rgb}{0.56,0,0}
\DefineVerbatimEnvironment{Sinput}{Verbatim}
{formatcom={\color{Sinput}},fontsize=\footnotesize, baselinestretch=0.75}
\DefineVerbatimEnvironment{Soutput}{Verbatim}
{formatcom={\color{Soutput}},fontsize=\footnotesize, baselinestretch=0.75}
%
% Rlogo
%
\newcommand{\Rlogo}{\protect\includegraphics[height=1.8ex,keepaspectratio]{../figs/Rlogo.pdf}}
%
% Shortcut for seqinR:
%
\newcommand{\seqinr}{\texttt{seqin\bf{R}}}
\newcommand{\Seqinr}{\texttt{Seqin\bf{R}}}
\fvset{fontsize= \scriptsize}
%
% R output options and libraries to be loaded.
%
%
%  Sweave Options
%
% Put all figures in the fig folder and start the name with current file name.
% Do not produce EPS files
%


\maketitle
\tableofcontents
% BEGIN - DO NOT REMOVE THIS LINE



%%%%%%%%%%%%%%%%%%%%%%%
\section{Introduction}
%%%%%%%%%%%%%%%%%%%%%%%

This chapter is under development and not complete.

%%%%%%%%%%%%%%%%%%%%%%%%%%%%
\section{System requirement}
%%%%%%%%%%%%%%%%%%%%%%%%%%%%

The socket server will build under a number of common Unix and Unix-alike
platforms.  In addition, binary distributions are available for the most
common Linux distributions and for Mac OS X  (XXX to be done XXX ). If you wish
to install the server from sources, or if you want to build an ACNUC database,
you will need several tools: programs are written in C thus you will need a
means of compiling C (as gcc compilation tools For Linux or Unix , Apple
Developer Tools  for MacOSX). You need as well library zlib and sockets
(standards on linux and unix). Basically if you are installing \Rlogo{} from the
sources, you should be able to build a ACNUC socket server.

%%%%%%%%%%%%%%%%%%%%%%%%%%%%%%%%%%%%%%%%%%%%%%%%%%%%%
\section{Technical description of the racnucd daemon}
%%%%%%%%%%%%%%%%%%%%%%%%%%%%%%%%%%%%%%%%%%%%%%%%%%%%%
Technical information about the acnuc socket server is available at this url:
\url{http://pbil.univ-lyon1.fr/databases/acnuc/racnucd.html}.


\section{ACNUC remote access protocol}
Description of the socket communication protocol with acnuc is availble at this url:
\url{http://pbil.univ-lyon1.fr/databases/acnuc/remote_acnuc.html}

%%%%%%%%%%%%%%%%%%%%%%%%%%%%%%%%%%%%%%%%%%%%%%%%%%%%%%%%%%%%%%
\section{Setting a local ACNUC database to be queried by the server}
%%%%%%%%%%%%%%%%%%%%%%%%%%%%%%%%%%%%%%%%%%%%%%%%%%%%%%%%%%%%%%

First of all yo need an ACNUC database, built by yourself or  downloaded from the PBIL ftp server.
An ACNUC database is composed of two sets of files:
\begin{enumerate}
	\item  the acnuc index files.
	\item  the database files (\textit{i.e.} flat files in EMBL/GenBank or SwissProt format).
\end{enumerate}

These two sets will be located  in the \textit{acnuc} and  \textit{gcgacnuc} directories  respectively.
(Directories \textit{acnuc} and  \textit{gcgacnuc} may be identical however it is prefered to use two distict directories.)

An example of an ACNUC database is available on the PBIL ftp server  at this url:
\url{ftp://pbil.univ-lyon1.fr/pub/seqinr/demoacnuc/acnucdatabase.tar.Z}.
 
You may install the database as it follows:
Let ACNUC\_HOME be the base directory for ACNUC intallation.

\begin{Schunk}
\begin{Sinput}
 dir.create("../ACNUC_HOME", showWarning = FALSE)
\end{Sinput}
\end{Schunk}

\begin{itemize}
\item Dowload the ACNUC database.

% Mettre eval=F apres premier download
\begin{Schunk}
\begin{Sinput}
 download.file("ftp://pbil.univ-lyon1.fr/pub/seqinr/demoacnuc/acnucdatabase.tar.Z", 
     destfile = "acnucdatabase.tar.Z")
\end{Sinput}
\end{Schunk}

\item Copy the  acnucdatabase.tar.Z file  into ACNUC\_HOME.

\begin{Schunk}
\begin{Sinput}
 system("cp acnucdatabase.tar.Z ../ACNUC_HOME")
\end{Sinput}
\end{Schunk}

\item Uncompress and untar the \texttt{acnucdatabase.tar.Z} file 

\begin{Schunk}
\begin{Sinput}
 pwd <- getwd()
 setwd("../ACNUC_HOME")
 system("gunzip -f acnucdatabase.tar.Z")
 system("tar -xvf acnucdatabase.tar")
 system("rm -f acnucdatabase.tar")
 setwd(pwd)
\end{Sinput}
\end{Schunk}

\end{itemize}
Now you sould get the following directories:
\begin{verbatim}
ACNUC_HOME/acnucdatabase/index
ACNUC_HOME/acnucdatabase/flat_files
\end{verbatim}

\begin{Schunk}
\begin{Sinput}
 dir("../ACNUC_HOME")
\end{Sinput}
\begin{Soutput}
[1] "flat_files" "index"     
\end{Soutput}
\end{Schunk}
which are the   \textit{acnuc} and  \textit{gcgacnuc} directories  respectively.

This database contains the complete genome of \textit{Escherichia coli K12 W3110} and
\textit{Saccharomyces cerevesiae}.

%%%%%%%%%%%%%%%%%%%%%%%%%%%%%%%%%%%%%%%%%%%%%%%%%%%%%%%%%%
\section{Build the ACNUC sockets server from the sources.}
%%%%%%%%%%%%%%%%%%%%%%%%%%%%%%%%%%%%%%%%%%%%%%%%%%%%%%%%%%

Once you have a local  ACNUC database available on your server you need to install the sockets server.

\subsection{Download the sources.}
The code source of the racnucd server is available on the PBIL server  at this url:
\begin{verbatim}
http://pbil.univ-lyon1.fr/databases/acnuc/racnucd.html
\end{verbatim}
Alternatively you can download directly  the source from the ftp at:
\begin{verbatim}
ftp://pbil.univ-lyon1.fr/pub/acnuc/unix/racnucd.tar
\end{verbatim}

\subsection{Build the ACNUC sockets server.}
You may install the racnucd server as it follows:
let RACNUCD\_HOME be the base directory for the socket.

\begin{Schunk}
\begin{Sinput}
 dir.create("../RACNUCD_HOME", showWarning = FALSE)
\end{Sinput}
\end{Schunk}

\begin{itemize}
\item Dowload the \texttt{racnucd.tar} file.

% Mettre eval=F apres premier download
\begin{Schunk}
\begin{Sinput}
 download.file("ftp://pbil.univ-lyon1.fr/pub/acnuc/unix/racnucd.tar", 
     destfile = "racnucd.tar")
\end{Sinput}
\end{Schunk}

\item Copy the \texttt{racnucd.tar} file into RACNUCD\_HOME.

\begin{Schunk}
\begin{Sinput}
 system("cp racnucd.tar ../RACNUCD_HOME")
\end{Sinput}
\end{Schunk}

\item Untar the \texttt{racnucd.tar} file 

\begin{Schunk}
\begin{Sinput}
 setwd("../RACNUCD_HOME")
 system("tar -xvf racnucd.tar")
 system("rm -f racnucd.tar")
 setwd(pwd)
\end{Sinput}
\end{Schunk}

\end{itemize}
Now you sould get the following directory:
\begin{verbatim}
RACNUCD_HOME/racnucd
\end{verbatim}

\begin{Schunk}
\begin{Sinput}
 dir("../RACNUCD_HOME")
\end{Sinput}
\begin{Soutput}
[1] "racnucd"
\end{Soutput}
\end{Schunk}

Go into RACNUCD\_HOME/racnucd/ and type;
\begin{verbatim}
make
\end{verbatim}
This should create the \textbf{racnucd} executable.

\begin{Schunk}
\begin{Sinput}
 setwd("../RACNUCD_HOME/racnucd")
 system("make")
 dir(pattern = "racnucd")
\end{Sinput}
\begin{Soutput}
[1] "racnucd"     "racnucd.ini" "racnucd.log"
\end{Soutput}
\begin{Sinput}
 setwd(pwd)
\end{Sinput}
\end{Schunk}

\subsection{Setting the ACNUC sockets server.}
The server is configured by several parameters described in a configuration file \textbf{racnuc.ini}.
The \textbf{racnucd.ini} file is structued as follows:

\begin{verbatim}
port=5558
maxtime=8000
known_db_file=knowndbs
db_env_names=dbplaces
\end{verbatim}

\begin{Schunk}
\begin{Sinput}
 ini <- readLines("../RACNUCD_HOME/racnucd/racnucd.ini")
 cat(ini, sep = "\n")
\end{Sinput}
\begin{Soutput}
port=5558
maxtime=8000
known_db_file=knowndbs
db_env_names=dbplaces
\end{Soutput}
\end{Schunk}

\begin{itemize}
\item \textbf{knowndbs} is a file containing the list of available databases
\item \textbf{dbplaces} is a file containing the path of the available databases
\end{itemize}

The \textbf{knowndbs} contains:
\begin{verbatim}
embl | on |    | EMBL sequence data library | 
swissprot   | on |  | UniProt |
\end{verbatim}

\begin{Schunk}
\begin{Sinput}
 knowndbs <- readLines("../RACNUCD_HOME/racnucd/knowndbs")
 cat(knowndbs, sep = "\n")
\end{Sinput}
\begin{Soutput}
embl | on |    | EMBL sequence data library | 
swissprot   | on |  | UniProt |
\end{Soutput}
\end{Schunk}

Each line defines a database,  the four fields indicating respectively the name
 of the database, its status  (\textit{on} or \textit{off}), a tag and a short description.

The \textbf{dbplaces} contains:
\begin{verbatim}
setenv  swissprot       '/Users/mgouy/Documents/acnuc/petite/swissprot /Users/mgouy/Documents/acnuc/petite/swissprot'
setenv  embl    '/Users/mgouy/Documents/acnuc/petite/embl /Users/mgouy/Documents/acnuc/petite/embl'
\end{verbatim}

\begin{Schunk}
\begin{Sinput}
 dbplaces <- readLines("../RACNUCD_HOME/racnucd/dbplaces")
 cat(dbplaces, sep = "\n")
\end{Sinput}
\begin{Soutput}
#defines location of acnuc databases index files and flat files

setenv 	swissprot 	'/Users/mgouy/Documents/acnuc/petite/swissprot /Users/mgouy/Documents/acnuc/petite/swissprot'
setenv 	embl 	'/Users/mgouy/Documents/acnuc/petite/embl /Users/mgouy/Documents/acnuc/petite/embl'
\end{Soutput}
\end{Schunk}

Each line set the acnuc and gcgacnuc variables for each   database.

Then you should set the files \textbf{knowndbs} and \textbf{dbplaces} according to your installation.
Let's call the database you installed previously \textit{demoacnuc}.
Modify  the \textbf{knowndbs} as follows:
\begin{verbatim}
demoacnuc | on |    | Demo Database | 
\end{verbatim}

\begin{Schunk}
\begin{Sinput}
 writeLines("demoacnuc | on |    | Demo Database | ", "../RACNUCD_HOME/racnucd/knowndbs")
 knowndbs <- readLines("../RACNUCD_HOME/racnucd/knowndbs")
 cat(knowndbs, sep = "\n")
\end{Sinput}
\begin{Soutput}
demoacnuc | on |    | Demo Database | 
\end{Soutput}
\end{Schunk}

and modify  the \textbf{dbplaces} as follows:

\begin{verbatim}
setenv  demoacnuc       'ACNUC_HOME/acnucdatabase/index ACNUC_HOME/acnucdatabase/flat_files'
\end{verbatim}

\begin{Schunk}
\begin{Sinput}
 indexpath <- normalizePath("../ACNUC_HOME/index")
 ffpath <- normalizePath("../ACNUC_HOME/flat_files")
 newdb <- paste("setenv demoacnuc '", indexpath, " ", ffpath, 
     "'", sep = "", collapse = "")
 writeLines(newdb, "../RACNUCD_HOME/racnucd/dbplaces")
 dbplaces <- readLines("../RACNUCD_HOME/racnucd/dbplaces")
 cat(dbplaces, sep = "\n")
\end{Sinput}
\begin{Soutput}
setenv demoacnuc '/Users/lobry/seqinr/inst/doc/src/ACNUC_HOME/index /Users/lobry/seqinr/inst/doc/src/ACNUC_HOME/flat_files'
\end{Soutput}
\end{Schunk}

Finaly, in the RACNUCD\_HOME/racnucd/ directory, type
\begin{verbatim}
./racnucd racnucd.ini
\end{verbatim}

The server should launch and say:
\begin{verbatim}

*******************************************************
Start of remote acnuc server : Tue Oct 16 15:46:16 2007

\end{verbatim}


\begin{Schunk}
\begin{Sinput}
 setwd("../RACNUCD_HOME/racnucd")
 system("./racnucd racnucd.ini > racnucd.log &")
 Sys.sleep(1)
 system("ps | grep racnucd", intern = TRUE)
\end{Sinput}
\begin{Soutput}
[1] "10482  p1  S+     0:00.01 ./racnucd racnucd.ini"  
[2] "10486  p1  S+     0:00.03 ./racnucd racnucd.ini"  
[3] "10501  p1  S+     0:00.01 sh -c ps | grep racnucd"
[4] "10503  p1  R+     0:00.00 grep racnucd"           
\end{Soutput}
\begin{Sinput}
 cat(readLines("racnucd.log"), sep = "\n")
\end{Sinput}
\begin{Soutput}
*******************************************************
Start of remote acnuc server : Wed Oct 17 20:39:34 2007
\end{Soutput}
\begin{Sinput}
 setwd(pwd)
\end{Sinput}
\end{Schunk}

The server is now ready.

\subsection{Using seqinR to query your local socket server.}
Launch \Rlogo{}, load the \texttt{seqinr} package  and type

\begin{Schunk}
\begin{Sinput}
choosebank(host="my_machine.domaine",info=T)
\end{Sinput}
\end{Schunk}
for example:

\begin{Schunk}
\begin{Sinput}
choosebank(host="semle.univ-lyon1.fr",info=T)
\end{Sinput}

\begin{Schunk}
\begin{Sinput}
 library(seqinr)
 hostname <- system("hostname", intern = TRUE)
 hostname
\end{Sinput}
\begin{Soutput}
[1] "sueoka.local"
\end{Soutput}
\begin{Sinput}
 choosebank(host = hostname, info = TRUE)
\end{Sinput}
\begin{Soutput}
       bank status
1 demoacnuc     on
                                                                 info
1 ACNUC database example. (September 2007) Last Updated: Oct 15, 2007
\end{Soutput}
\end{Schunk}

You should obtain something like:

\begin{Soutput}
       bank status
1 demoacnuc     on
                                                                 info
1 ACNUC database example. (September 2007) Last Updated: Oct 15, 2007
\end{Soutput}
\end{Schunk}

You can now query the database. For example:

\begin{Schunk}
\begin{Sinput}
 choosebank(bank = "demoacnuc", host = hostname)
 query("mylist", "k=rib@ prot@")
 mylist$nelem
\end{Sinput}
\begin{Soutput}
[1] 39
\end{Soutput}
\begin{Sinput}
 sapply(mylist$req, getName)
\end{Sinput}
\begin{Soutput}
 [1] "AP009048.PE25"   "AP009048.PE405"  "AP009048.PE830"  "AP009048.PE3223"
 [5] "AP009048.PE3465" "AP009048.PE3466" "AP009048.PE3516" "U00091.PE38"    
 [9] "U00093.PE119"    "U00093.PE123"    "U00094.PE65"     "U00094.PE87"    
[13] "U00094.PE262"    "U00094.PE393"    "U00094.PE400"    "X59720.PE36"    
[17] "Y13134.PE91"     "Y13134.PE272"    "Y13135.PE271"    "Y13137.PE286"   
[21] "Y13138.PE70"     "Y13138.PE198"    "Y13138.PE280"    "Y13139.PE53"    
[25] "Y13139.PE110"    "Y13139.PE316"    "Y13140.PE89"     "Z47047.PE177"   
[29] "Z47047.PE180"    "Z71256.PE178"    "Z71256.PE289"    "Z71256.PE313"   
[33] "Z71256.PE317"    "Z71256.PE534"    "Z71256.PE637"    "Z71256.PE694"   
[37] "Z71257.PE43"     "Z71257.PE75"     "Z71257.PE263"   
\end{Soutput}
\end{Schunk}

%%%%%%%%%%%%%%%%%%%%%%%%%%%%%%%%%%%%%%%%%%%%%%%%%%%%%%%%%%%%%
\section{ACNUC sockets server binaries distributions. (under development)}
%%%%%%%%%%%%%%%%%%%%%%%%%%%%%%%%%%%%%%%%%%%%%%%%%%%%%%%%%%%%%

Binaries are available for several machines: 

\begin{itemize}
	\item  Unix sunOS
	\item  Linux RedHat
	\item Darwin Mac OS
\end{itemize}

You can download the different distribution at this address: 
(XXX to be done XXX ).

Then you can install the socket server :

Bla.. blah blah




%%%%%%%%%%%%%%%%%%%%%%%%%%%%%%%%%%%%%%%%%%%%%%%%%%%%%%%%%%%%%
\section{Building your own  ACNUC database. (under development)}
%%%%%%%%%%%%%%%%%%%%%%%%%%%%%%%%%%%%%%%%%%%%%%%%%%%%%%%%%%%%%
One of the interest of a local server is to be able use your own ACNUC
database.

 \subsection{System requirements.}
 
ACNUC runs on UNIX systems, Windows NT/95/98/00/XP, MacOS 9/X. Tested UNIX
systems include SUN, Linux on PC, IBM RS/6000, Silicon Graphics, DEC Alpha,
Fujitsu UXP/M. ACNUC should be easily installed on any other UNIX platforms. 

\subsection{Database requirements.}

ACNUC database are build from flat files in several possible format : EMBL, Genbank or
SwissProt.

\subsection{Download the ACNUC dababase management tools.}
 
\subsection{Install the ACNUC dababase management tools.}

\subsection{Database building}

\section{Misc} 
%%%%%%%%%%%%%

Other tools for acnuc

Several powerful tools dedicated to query ACNUC databases are available. 

The programs \textbf{query} and \textbf{query\_win} allow to query an ACNUC database according to the same
criteria than described  in seqinR. It allows as well several functionality to extact biological data.
\textbf{query\_win} is a graphical version of query.
\textbf{query} is an on-line version which allows to query and ACNUC database through scripts.
Both \textbf{query} and  \textbf{query\_win} are available as a
\textit{client} or a \textit{local} application. More information on these programs can be found at:
\begin{verbatim}
http://pbil.univ-lyon1.fr/software/query_win.html
\end{verbatim}. 




\section{Citation} 
%%%%%%%%%%%%%

If you use query\_win or query in a published work, please cite the following reference:
Gouy M, Gautier C, Attimonelli M, Lanave C, di Paola G. (1985)
 ACNUC--a portable retrieval system for nucleic acid sequence databases: 
 logical and physical designs and usage.. CABIOS, 1, 167-172. 



\section{Session Informations}

This part was compiled under the following \Rlogo{}~environment:

\begin{itemize}
  \item R version 2.6.0 (2007-10-03), \verb|i386-apple-darwin8.10.1|
  \item Locale: \verb|C|
  \item Base packages: base, datasets, grDevices, graphics, methods,
    stats, utils
  \item Other packages: MASS~7.2-36, ade4~1.4-4, ape~2.0-1,
    gee~4.13-13, lattice~0.16-5, nlme~3.1-85, seqinr~1.1-3,
    xtable~1.5-1
  \item Loaded via a namespace (and not attached): grid~2.6.0,
    rcompgen~0.1-15
\end{itemize}
There were two compilation steps:

\begin{itemize}
  \item \Rlogo{} compilation time was: Wed Oct 17 20:39:35 2007
  \item \LaTeX{} compilation time was: \today
\end{itemize}

% END - DO NOT REMOVE THIS LINE

%%%%%%%%%%%%  BIBLIOGRAPHY %%%%%%%%%%%%%%%%%
\clearpage
\addcontentsline{toc}{section}{References}
\bibliographystyle{plain}
\bibliography{../config/book}
\end{document}
