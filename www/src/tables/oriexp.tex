\begin{table}[ht]
\begin{center}
\tiny
\begin{tabular}{lllllll}
\hline \hline
Species & Strain & Accession & Location & Flanking genes& Method & Reference\\
\hline
\textit{Borrelia burgdorferi} & B31 & NC\_001318 & 458,037..458276 & \textit{dnaA}..\textit{dnaN}& nascent DNA & \cite{PicardeauM1999}\\
\textit{Caulobacter crescentus} & CB15 & NC\_002696 &  & \textit{rrnA} & PFGE synch cell & \cite{DingwallA1989}\\
\textit{Enterobacter aerogenes} & SD1 & NA & NA & \textit{asnA}..\textit{uncB}& ARS in \textit{E. coli} & \cite{HardingNE1982}\\
\textit{Erwinia carotovora} & EC153 & NA & NA & \textit{asnA}& Method & \cite{TakedaY1982}\\

\textit{Halobacterium} sp. & NRC-1 & NC\_002607 & 1,807,230 & \textit{orc7}& ARS & \cite{BerquistBR2003}\\
\textit{Haloferax volcanii} &  & NA &  & &  & \cite{NoraisC2007}\\

\textit{Helicobacter pylori} & J99 & NC\_000921& 1,559,163..1,559,314& \textit{dnaA}..\textit{jhp1418}& DnaA binding & \cite{ZawilakA2001}\\
\textit{Klebsiella pneumoniae} & M51 & NA & NA & \textit{asnA}..\textit{uncB}& ARS in \textit{E. coli} & \cite{HardingNE1982}\\
\textit{Pseudomonas aeruginosa} & PAO2003 & NA & NA & \textit{rpmH}..\textit{dnaA}& ARS in \textit{Pseudomonas} & \cite{YeeTW1990}\\
\textit{Pseudomonas putida} & KT2440 & NC\_002947 & 8,947..9,541 & \textit{rpmH}..\textit{dnaA}& ARS in \textit{Pseudomonas} & \cite{YeeTW1990}\\
\textit{Pyrococcus abyssi} & GE5 & NC\_000868 & 122,701..123,499 & \textit{orc1}..\textit{mtaP}& many & \cite{MyllykallioH2000, MatsunagaF2001, MatsunagaF2003}\\
\textit{Salmonella thyphimurium} & SA970 & NA & NA & \textit{uncB}..\textit{asn} & ARS in \textit{E. coli} & \cite{ZyskindJW1979, ZyskindJW1980}\\
\textit{Sinorhizobium meliloti} & 1021 & NC\_003047 & 1..477 & \textit{hemE}..\textit{Y02793} & ARS in \textit{Sinorhizobium} & \cite{SibleyCD2006}\\
\textit{Streptomyces lividans} & TK21 $\leftarrow$ 66 & NA & NA & NA & ARS in \textit{S. coelicolor} A3(2) & \cite{ZakrezewskaCzerwinskaJ1992} \\

\textit{Sulfolobus acidocaldarius} & DSM 639 & NC\_007181  &  & & MF analysis& \cite{LundgrenM2004}\\


\textit{Sulfolobus solfataricus} & P2 & NC\_002754 & 221,077 & \textit{cdc6-1}& Synch. & \cite{RobinsonNP2004, LundgrenM2004}\\
 &  &  & 2,009,385 & \textit{cdc6-3}& & \\



\textit{Vibrio harveyi} & B392 & NA & NA & \textit{gid}..15,5& ARS in \textit{E. coli} & \cite{ZyskindJW1983}\\
\hline \hline
\end{tabular}
\caption{Replication origins in bacteria with experimental evidence}
\label{oriexp}
\end{center}
\end{table}
\normalsize