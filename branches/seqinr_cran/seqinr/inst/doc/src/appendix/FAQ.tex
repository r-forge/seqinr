\documentclass{article}
\input{../config/commontex}

\title{FAQ: Frequently Asked Questions}
\author{Lobry, J.R.}

\usepackage{/Library/Frameworks/R.framework/Resources/share/texmf/Sweave}
\begin{document}
%
% To change the R input/output style:
%
\definecolor{Soutput}{rgb}{0,0,0.56}
\definecolor{Sinput}{rgb}{0.56,0,0}
\DefineVerbatimEnvironment{Sinput}{Verbatim}
{formatcom={\color{Sinput}},fontsize=\footnotesize, baselinestretch=0.75}
\DefineVerbatimEnvironment{Soutput}{Verbatim}
{formatcom={\color{Soutput}},fontsize=\footnotesize, baselinestretch=0.75}
%
% Rlogo
%
\newcommand{\Rlogo}{\protect\includegraphics[height=1.8ex,keepaspectratio]{../figs/Rlogo.pdf}}
%
% Shortcut for seqinR:
%
\newcommand{\seqinr}{\texttt{seqin\bf{R}}}
\newcommand{\Seqinr}{\texttt{Seqin\bf{R}}}
\fvset{fontsize= \scriptsize}
%
% R output options and libraries to be loaded.
%
%
%  Sweave Options
%
% Put all figures in the fig folder and start the name with current file name.
% Do not produce EPS files
%


\maketitle
% BEGIN - DO NOT REMOVE THIS LINE

\section{How do I compute a score on my sequences?}

In the example below we want to compute the G+C content in third codon
positions for complete ribosomal CDS from \textit{Escherichia coli}:

\begin{Schunk}
\begin{Sinput}
 choosebank("emblTP")
 query("ecribo", "sp=escherichia coli ET t=cds ET k=ribosom@ ET NO k=partial")
 myseqs <- sapply(ecribo$req, getSequence)
 (gc3 <- sapply(myseqs, GC3))
\end{Sinput}
\begin{Soutput}
 [1] 0.4946237 0.6046512 0.5000000 0.6194030 0.5772727 0.4838710 0.5980066
 [8] 0.4974359 0.5031250 0.4324324 0.5000000 0.5113636 0.5290520 0.6142857
[15] 0.4904762 0.5714286 0.6191860 0.5906040 0.4880000 0.4880000 0.4946237
[22] 0.6046512 0.5000000 0.3522727 0.5076923 0.4343434 0.6194030 0.5522388
[29] 0.6104651 0.5661157 0.4946237 0.4946237 0.6079734 0.5000000 0.6343284
[36] 0.4659091 0.5789474 0.4946237 0.5000000 0.4974359 0.5689655 0.4611111
[43] 0.4611111 0.5303030 0.5303030 0.4482759 0.4201681 0.5915493 0.5000000
[50] 0.3829787 0.4519231 0.4302326 0.5696203 0.4285714 0.5689655 0.5000000
[57] 0.5224417 0.5661157 0.6057692 0.4444444 0.4659091 0.4130435 0.4946237
[64] 0.5661157 0.4946237 0.5680272
\end{Soutput}
\end{Schunk}

At the amino-acid level, we may get an estimate of the isoelectric point of
the proteins this way:

\begin{Schunk}
\begin{Sinput}
 sapply(sapply(myseqs, getTrans), computePI)
\end{Sinput}
\begin{Soutput}
 [1]  6.624309  7.801329 10.864793  5.931989  7.830476  6.624309  7.801329
 [8]  9.203410  9.826485  5.674672  7.154423  6.060457  6.313741  5.571446
[15]  9.435422  4.310747  6.145496  4.876054 11.006424 10.876041  6.624309
[22]  7.801329 10.864793  9.346289  9.203410  5.877050  5.931989  9.934988
[29]  5.920490  6.612505  6.624309  6.624309  7.801329 10.864793  5.931989
[36] 11.182505  9.598944  6.624309 10.864793  9.203410 11.031938  5.858421
[43]  5.858421 11.777511 11.777516 10.619175 11.365738  9.460987 10.864793
[50] 13.002381  9.845859 10.584868 11.421252 10.248325 11.031938 10.402075
[57]  4.863862  6.612505  9.681066 11.150310 11.182499 11.043607  6.624309
[64]  6.612505  6.624309  4.310747
\end{Soutput}
\end{Schunk}

Note that some pre-defined vectors to compute linear forms on sequences are
available in the \texttt{EXP} data.

As a matter of convenience, you may encapsulate the computation of your favorite score 
within a function this way:

\begin{Schunk}
\begin{Sinput}
 GC3m <- function(list, ind = 1:list$nelem) sapply(sapply(list$req[ind], 
     getSequence), GC3)
 GC3m(ecribo)
\end{Sinput}
\begin{Soutput}
 [1] 0.4946237 0.6046512 0.5000000 0.6194030 0.5772727 0.4838710 0.5980066
 [8] 0.4974359 0.5031250 0.4324324 0.5000000 0.5113636 0.5290520 0.6142857
[15] 0.4904762 0.5714286 0.6191860 0.5906040 0.4880000 0.4880000 0.4946237
[22] 0.6046512 0.5000000 0.3522727 0.5076923 0.4343434 0.6194030 0.5522388
[29] 0.6104651 0.5661157 0.4946237 0.4946237 0.6079734 0.5000000 0.6343284
[36] 0.4659091 0.5789474 0.4946237 0.5000000 0.4974359 0.5689655 0.4611111
[43] 0.4611111 0.5303030 0.5303030 0.4482759 0.4201681 0.5915493 0.5000000
[50] 0.3829787 0.4519231 0.4302326 0.5696203 0.4285714 0.5689655 0.5000000
[57] 0.5224417 0.5661157 0.6057692 0.4444444 0.4659091 0.4130435 0.4946237
[64] 0.5661157 0.4946237 0.5680272
\end{Soutput}
\begin{Sinput}
 GC3m(ecribo, 1:10)
\end{Sinput}
\begin{Soutput}
 [1] 0.4946237 0.6046512 0.5000000 0.6194030 0.5772727 0.4838710 0.5980066
 [8] 0.4974359 0.5031250 0.4324324
\end{Soutput}
\end{Schunk}

\section{Why do I have not exactly the same G+C content as in \texttt{codonW}?}

This question was raised (and solved) by Oliver Clay in an e-mail (23-AUG-2006).
The program \texttt{codonW} was written in C as part of Jonh Peden's PhD thesis 
on Codon Usage \cite{codonW} and is available at \url{http://codonw.sourceforge.net/}.
The reason for the small differences in G+C content between the two programs is
that the default behavior in \texttt{codonW} is to remove the stop codon before
computations. Here is one way of removing the stop codon under \Rlogo{}:

\begin{Schunk}
\begin{Sinput}
 gc3nos <- sapply(myseqs, function(s) GC3(s[1:(length(s) - 
     3)]))
\end{Sinput}
\end{Schunk}

As compared with the previous result, the difference is small but visible:

\begin{Schunk}
\begin{Sinput}
 plot(x = gc3, y = gc3nos, las = 1, main = "Stop codon removal effect on G+C content\nin third codon positions", 
     xlab = "With stop codon", ylab = "Stop codons removed")
 abline(c(0, 1))
\end{Sinput}
\end{Schunk}
\includegraphics{../figs/FAQ-stopcodonremovaleffect}

\section{How do I get a sequence from its name?}

This question is adapted from an e-mail (22 Jun 2006) by Gang Xu.
I know that the UniProt (SwissProt) entry of my protein is \texttt{P08758},
if I know its name\footnote{
More exactly, this is the accession number. Sequence names are not stable over time,
it's always better to use the accession numbers. 
}, how can I get the sequence?

\begin{Schunk}
\begin{Sinput}
 choosebank("swissprot")
 query("myprot", "AC=P08758")
 getSequence(myprot$req[[1]])
\end{Sinput}
\begin{Soutput}
  [1] "A" "Q" "V" "L" "R" "G" "T" "V" "T" "D" "F" "P" "G" "F" "D" "E" "R" "A"
 [19] "D" "A" "E" "T" "L" "R" "K" "A" "M" "K" "G" "L" "G" "T" "D" "E" "E" "S"
 [37] "I" "L" "T" "L" "L" "T" "S" "R" "S" "N" "A" "Q" "R" "Q" "E" "I" "S" "A"
 [55] "A" "F" "K" "T" "L" "F" "G" "R" "D" "L" "L" "D" "D" "L" "K" "S" "E" "L"
 [73] "T" "G" "K" "F" "E" "K" "L" "I" "V" "A" "L" "M" "K" "P" "S" "R" "L" "Y"
 [91] "D" "A" "Y" "E" "L" "K" "H" "A" "L" "K" "G" "A" "G" "T" "N" "E" "K" "V"
[109] "L" "T" "E" "I" "I" "A" "S" "R" "T" "P" "E" "E" "L" "R" "A" "I" "K" "Q"
[127] "V" "Y" "E" "E" "E" "Y" "G" "S" "S" "L" "E" "D" "D" "V" "V" "G" "D" "T"
[145] "S" "G" "Y" "Y" "Q" "R" "M" "L" "V" "V" "L" "L" "Q" "A" "N" "R" "D" "P"
[163] "D" "A" "G" "I" "D" "E" "A" "Q" "V" "E" "Q" "D" "A" "Q" "A" "L" "F" "Q"
[181] "A" "G" "E" "L" "K" "W" "G" "T" "D" "E" "E" "K" "F" "I" "T" "I" "F" "G"
[199] "T" "R" "S" "V" "S" "H" "L" "R" "K" "V" "F" "D" "K" "Y" "M" "T" "I" "S"
[217] "G" "F" "Q" "I" "E" "E" "T" "I" "D" "R" "E" "T" "S" "G" "N" "L" "E" "Q"
[235] "L" "L" "L" "A" "V" "V" "K" "S" "I" "R" "S" "I" "P" "A" "Y" "L" "A" "E"
[253] "T" "L" "Y" "Y" "A" "M" "K" "G" "A" "G" "T" "D" "D" "H" "T" "L" "I" "R"
[271] "V" "M" "V" "S" "R" "S" "E" "I" "D" "L" "F" "N" "I" "R" "K" "E" "F" "R"
[289] "K" "N" "F" "A" "T" "S" "L" "Y" "S" "M" "I" "K" "G" "D" "T" "S" "G" "D"
[307] "Y" "K" "K" "A" "L" "L" "L" "L" "C" "G" "E" "D" "D"
\end{Soutput}
\end{Schunk}

\section{Session Informations}

This part was compiled under the following \Rlogo{}~environment:

\begin{itemize}
  \item Version 2.3.1 (2006-06-01), \verb|powerpc-apple-darwin8.6.0|
  \item Base packages: base, datasets, grDevices, graphics, methods,
    stats, utils
  \item Other packages: MASS~7.2-27.1, ade4~1.4-1, ape~1.8-2,
    gee~4.13-10, lattice~0.13-8, nlme~3.1-73, seqinr~1.0-6,
    xtable~1.3-0
\end{itemize}
% END - DO NOT REMOVE THIS LINE

%%%%%%%%%%%%  BIBLIOGRAPHY %%%%%%%%%%%%%%%%%
\clearpage
\addcontentsline{toc}{section}{References}
\bibliographystyle{plain}
\bibliography{../config/book}
\end{document}
